\documentclass{article}

\usepackage[french]{babel}
\usepackage[utf8]{inputenc}
\usepackage{amsmath}
\usepackage{url}

%%%%%%%%%%%%%%%% Lengths %%%%%%%%%%%%%%%%
\setlength{\textwidth}{15.5cm}
\setlength{\evensidemargin}{0.5cm}
\setlength{\oddsidemargin}{0.5cm}

\begin{document}

%%%%%%%%%%%%%%%% Main part %%%%%%%%%%%%%%%%
\section{Méthode du gradient conjugué}
\label{sec:gradient_conjugue}

[...]
\vskip 1mm ~

Ceci est un brouillon (pour l'instant)

N.B. : En mathlab, $r'$ est transposée de $r$.
\vskip 1mm ~
Cette méthode diffère de la méthode proposée en section~\ref{sec:decomposition_cholesky} car celle-ci est une méthode itérative.[gain en complexité?][systématique?]

[implémentation]

[implémentation avec poréconditionneur]

[commentaires?]

\end{document}
